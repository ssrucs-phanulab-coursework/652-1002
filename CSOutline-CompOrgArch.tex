%
\documentclass[
  notoc % Suppress Tufte style table of contents.
]{tufte-book}

% Required Tufte packages.
\usepackage{changepage} % or changepage
\usepackage{fancyhdr}
\usepackage{fontenc}
\usepackage{geometry}
\usepackage{hyperref}
\usepackage{natbib}
\usepackage{bibentry}
\usepackage{optparams}
\usepackage{paralist}
\usepackage{placeins}
\usepackage{ragged2e}
\usepackage{setspace}
\usepackage{textcase}
\usepackage{textcomp}
\usepackage{titlesec}
\usepackage{titletoc}
\usepackage{xcolor}
\usepackage{xifthen}

\geometry{paperheight=10in,paperwidth=7in,marginparwidth=30mm,marginparsep=2mm,bindingoffset=10mm,top=10mm,inner=8mm,outer=8mm,bottom=16mm,includehead,includemp}

% Tufte vs. Pandoc workaround.
% Issue: https://github.com/Tufte-LaTeX/tufte-latex/issues/64.
\renewcommand\allcapsspacing[1]{{\addfontfeature{LetterSpace=15}#1}}
\renewcommand\smallcapsspacing[1]{{\addfontfeature{LetterSpace=10}#1}}

% \setmainfont{TeX Gyre Pagella}
\usepackage[utf8]{inputenc}
\usepackage[T1]{fontenc}
\setmainfont{texgyrepagella}[
  Extension = .otf,
  UprightFont = *-regular,
  BoldFont = *-bold,
  ItalicFont = *-italic,
  BoldItalicFont = *-bolditalic,
]

\usepackage{fontspec}
\setmonofont{JuliaMono-Medium.ttf}[
    % Do not remove the trailing forward slash.
    Path = /home/runner/.julia/artifacts/45f34ceb7f1b7b67949715de56b123afeaa72e47/juliamono-0.045/,
    Contextuals = Alternate,
    Ligatures = NoCommon
]

\newfontface\JuliaMonoRegular{JuliaMono-Regular.ttf}[
    Path = /home/runner/.julia/artifacts/45f34ceb7f1b7b67949715de56b123afeaa72e47/juliamono-0.045/,
    Contextuals = Alternate,
    Ligatures = NoCommon
]

\newfontface\JuliaMonoBold{JuliaMono-Bold.ttf}[
    Path = /home/runner/.julia/artifacts/45f34ceb7f1b7b67949715de56b123afeaa72e47/juliamono-0.045/,
    Contextuals = Alternate,
    Ligatures = NoCommon
]



\DeclareRobustCommand{\href}[2]{#2\footnote{\url{#1}}}

\usepackage{float}
\floatplacement{figure}{H}

% Listings Julia syntax definition.
\input{/home/runner/.julia/packages/Books/odNoe/defaults/julia_listings.tex}

% Unicode support.
\input{/home/runner/.julia/packages/Books/odNoe/defaults/julia_listings_unicode.tex}

% Used by Pandoc.
\providecommand{\tightlist}{%
  \setlength{\itemsep}{0pt}\setlength{\parskip}{0pt}
}
\newcommand{\passthrough}[1]{#1}

\usepackage{longtable}
\usepackage{booktabs}
\usepackage{array}

% Source: Wandmalfarbe/pandoc-latex-template.

\definecolor{linkblue}{HTML}{117af2}
\usepackage{hyperref}
\hypersetup{
  colorlinks,
  citecolor=linkblue,
  linkcolor=linkblue,
  urlcolor=linkblue,
  linktoc=page, % Avoid Table of Contents being nearly completely blue.
  pdftitle={Computer Organization \& Architecture},
  pdfauthor={ภาณุ วราภรณ์; นักศึกษารหัส 65},
  pdflang={en-US},
  breaklinks=true,
  pdfcreator={LaTeX via Pandoc}%
}
\urlstyle{same} % disable monospaced font for URLs

\title{Computer Organization \& Architecture}
\author{\noindent{ภาณุ วราภรณ์}\\[3mm] \noindent{นักศึกษารหัส 65}\\[3mm] }
\date{}

% Re-enable section numbering which was disabled by tufte.
\setcounter{secnumdepth}{2}

% Fix captions for longtable.
% Thanks to David Carlisle at https://tex.stackexchange.com/a/183344/92217.
\makeatletter
\def\LT@makecaption#1#2#3{%
  \noalign{\smash{\hbox{\kern\textwidth\rlap{\kern\marginparsep
  \parbox[t]{\marginparwidth}{\vspace{12pt}%
\@tufte@caption@font \@tufte@caption@justification \noindent
   #1{#2: }\ignorespaces #3}}}}}}
\makeatother

% Doesn't seem to do anything.
\usepackage{float}
\floatplacement{figure}{H}
\floatplacement{table}{H}

% Reduce large spacing around sections.
\titlespacing*{\chapter}{0pt}{5pt}{20pt}
\titlespacing*{\section}{0pt}{2.5ex plus 1ex minus .2ex}{1.3ex plus .2ex}
\titlespacing*{\subsection}{0pt}{1.75ex plus 1ex minus .2ex}{1.0ex plus.2ex}

\titleformat{\chapter}%
  [hang]% shape
  {\normalfont\huge\itshape}% format applied to label+text
  {\huge\thechapter}% label
  {1em}% horizontal separation between label and title body
  {}% before the title body
  []% after the title body

% Reduce spacing in table of contents.
\usepackage{etoolbox}
\makeatletter
\pretocmd{\chapter}{\addtocontents{toc}{\protect\addvspace{-3\p@}}}{}{}
\pretocmd{\section}{\addtocontents{toc}{\protect\addvspace{-4\p@}}}{}{}
\pretocmd{\subsection}{\addtocontents{toc}{\protect\addvspace{-5\p@}}}{}{}
\makeatother

% Long texts are harder to read than tables.
% Therefore, we can reduce the font size of the table.
\AtBeginEnvironment{longtable}{\footnotesize}

% Some space between paragraphs is necessary because code blocks can output single line paragraphs.
\setlength\parskip{1em plus 0.1em minus 0.2em}

% For justified text.
\usepackage{ragged2e}

% tufte-book disables subsubsections by default.
% Got this definition back via `\show\subsubsection`.

\usepackage{amsfonts}
\usepackage{amssymb}
\usepackage{amsmath}
\usepackage{unicode-math}

% URL line breaks.
\usepackage{xurl}

% Probably doesn't hurt.
\usepackage{marginfix}




\begin{document}

\makeatletter
\thispagestyle{empty}
\vfill
{\Huge\bf
\noindent
\@title
}\\[1in]
{\Large
\noindent
\@author
}
\makeatother

\makeatletter
\newpage
\thispagestyle{empty}
\vfill
{\noindent
\begin{tabular}{l} ภาณุ วราภรณ์\\ สาขาวิชาวิทยาการคอมพิวเตอร์\\ \\ นักศึกษารหัส 65\\ สาขาวิชาวิทยาการคอมพิวเตอร์\\ \\ \end{tabular}
}
\vfill
{\small
\url{https://tba.githubpages.io}

Version: 2024-03-23

Creative Commons Attribution-NonCommercial-ShareAlike 4.0 International
}
\makeatother


% Don't remove this or authors will show up in header of every page.
\frontmatter
\mainmatter
\fancyfoot[C]{\url{https://github.com/johndoe/Book.jl}}

\setcounter{tocdepth}{1}
\tableofcontents

% Justify text.
\justifying

% parindent seems to be set from within another class too.
% it is really not useful here because it will also indent lines directly after
% code blocks. Which most of the times not useful.
\setlength{\parindent}{0pt}

\hypertarget{about}{%
\chapter*{About}\label{about}}
\addcontentsline{toc}{chapter}{About}

รายวิชานี้ศึกษาทฤษฎีเกี่ยวกับสถาปัตยกรรมคอมพิวเตอร์ที่มีการพัฒนามาและทิศทางในอนาคต
ครอบคลุมเนื้อหาที่เป็นพื้นฐานที่สำคัญและจำเป็นสำหรับใช้ประกอบการศึกษาในรายวิชาต่อไป
รายวิชานี้มีการศึกษาและฝึกปฎิบัติการคอมพิวเตอร์เพื่อให้นักศึกษาได้นำทฤษฎีไปปฎิบัติใช้ได้จริง และ
เพิมพูนทักษะสำหรับเตรียมความพร้อมในการไปประกอบวิชาชีพได้

\hypertarget{sec:intro}{%
\chapter{Introduction}\label{sec:intro}}

รายวิชานี้ศึกษาทฤษฎีเกี่ยวกับสถาปัตยกรรมคอมพิวเตอร์ที่มีการพัฒนามาและทิศทางในอนาคต
ครอบคลุมเนื้อหาที่เป็นพื้นฐานที่สำคัญและจำเป็นสำหรับใช้ประกอบการศึกษาในรายวิชาต่อไป
รายวิชานี้มีการศึกษาและฝึกปฎิบัติการคอมพิวเตอร์เพื่อให้นักศึกษาได้นำทฤษฎีไปปฎิบัติใช้ได้จริง และ
เพิมพูนทักษะสำหรับเตรียมความพร้อมในการไปประกอบวิชาชีพได้

\hypertarget{sec:cprog}{%
\chapter{C Programming Language}\label{sec:cprog}}

รายวิชานี้ศึกษาทฤษฎีเกี่ยวกับสถาปัตยกรรมคอมพิวเตอร์ที่มีการพัฒนามาและทิศทางในอนาคต
ครอบคลุมเนื้อหาที่เป็นพื้นฐานที่สำคัญและจำเป็นสำหรับใช้ประกอบการศึกษาในรายวิชาต่อไป
รายวิชานี้มีการศึกษาและฝึกปฎิบัติการคอมพิวเตอร์เพื่อให้นักศึกษาได้นำทฤษฎีไปปฎิบัติใช้ได้จริง และ
เพิมพูนทักษะสำหรับเตรียมความพร้อมในการไปประกอบวิชาชีพได้

\hypertarget{sec:moreon_cprog}{%
\chapter{More on C Programming}\label{sec:moreon_cprog}}

รายวิชานี้ศึกษาทฤษฎีเกี่ยวกับสถาปัตยกรรมคอมพิวเตอร์ที่มีการพัฒนามาและทิศทางในอนาคต
ครอบคลุมเนื้อหาที่เป็นพื้นฐานที่สำคัญและจำเป็นสำหรับใช้ประกอบการศึกษาในรายวิชาต่อไป
รายวิชานี้มีการศึกษาและฝึกปฎิบัติการคอมพิวเตอร์เพื่อให้นักศึกษาได้นำทฤษฎีไปปฎิบัติใช้ได้จริง และ
เพิมพูนทักษะสำหรับเตรียมความพร้อมในการไปประกอบวิชาชีพได้

\hypertarget{sec:cdebuggingtools}{%
\chapter{C Debugging Tools}\label{sec:cdebuggingtools}}

\begin{lstlisting}
รายวิชานี้ศึกษาทฤษฎีเกี่ยวกับสถาปัตยกรรมคอมพิวเตอร์ที่มีการพัฒนามาและทิศทางในอนาคต ครอบคลุมเนื้อหาที่เป็นพื้นฐานที่สำคัญและจำเป็นสำหรับใช้ประกอบการศึกษาในรายวิชาต่อไป รายวิชานี้มีการศึกษาและฝึกปฎิบัติการคอมพิวเตอร์เพื่อให้นักศึกษาได้นำทฤษฎีไปปฎิบัติใช้ได้จริง และ เพิมพูนทักษะสำหรับเตรียมความพร้อมในการไปประกอบวิชาชีพได้

1.GNU debugger (GDB): examining a program's runtime state
2.Valgrind: code profiling suite
    Memcheck analyze program's memory access to detect invalid memory usage, uninitialized memory usage, and memory leaks
\end{lstlisting}

\hypertarget{sec:debugging-with-GDB}{%
\section{Debugging with GDB}\label{sec:debugging-with-GDB}}

\begin{lstlisting}
การหาข้อผิดพลาดด้วย GDB
\end{lstlisting}

\hypertarget{sec:GDB-commands}{%
\section{GDB commands in detail}\label{sec:GDB-commands}}

\begin{lstlisting}
คำสั่งในการใช้งาน GDB
\end{lstlisting}

\hypertarget{sec:bidatarep}{%
\chapter{Binary and Data Representation}\label{sec:bidatarep}}

รายวิชานี้ศึกษาทฤษฎีเกี่ยวกับสถาปัตยกรรมคอมพิวเตอร์ที่มีการพัฒนามาและทิศทางในอนาคต
ครอบคลุมเนื้อหาที่เป็นพื้นฐานที่สำคัญและจำเป็นสำหรับใช้ประกอบการศึกษาในรายวิชาต่อไป
รายวิชานี้มีการศึกษาและฝึกปฎิบัติการคอมพิวเตอร์เพื่อให้นักศึกษาได้นำทฤษฎีไปปฎิบัติใช้ได้จริง และ
เพิมพูนทักษะสำหรับเตรียมความพร้อมในการไปประกอบวิชาชีพได้

\hypertarget{sec:neumancomparch}{%
\chapter{von Neuman Computer Architecture}\label{sec:neumancomparch}}

รายวิชานี้ศึกษาทฤษฎีเกี่ยวกับสถาปัตยกรรมคอมพิวเตอร์ที่มีการพัฒนามาและทิศทางในอนาคต
ครอบคลุมเนื้อหาที่เป็นพื้นฐานที่สำคัญและจำเป็นสำหรับใช้ประกอบการศึกษาในรายวิชาต่อไป
รายวิชานี้มีการศึกษาและฝึกปฎิบัติการคอมพิวเตอร์เพื่อให้นักศึกษาได้นำทฤษฎีไปปฎิบัติใช้ได้จริง และ
เพิมพูนทักษะสำหรับเตรียมความพร้อมในการไปประกอบวิชาชีพได้

\hypertarget{sec:diveintoassembly}{%
\chapter{Dive into Assembly}\label{sec:diveintoassembly}}

รายวิชานี้ศึกษาทฤษฎีเกี่ยวกับสถาปัตยกรรมคอมพิวเตอร์ที่มีการพัฒนามาและทิศทางในอนาคต
ครอบคลุมเนื้อหาที่เป็นพื้นฐานที่สำคัญและจำเป็นสำหรับใช้ประกอบการศึกษาในรายวิชาต่อไป
รายวิชานี้มีการศึกษาและฝึกปฎิบัติการคอมพิวเตอร์เพื่อให้นักศึกษาได้นำทฤษฎีไปปฎิบัติใช้ได้จริง และ
เพิมพูนทักษะสำหรับเตรียมความพร้อมในการไปประกอบวิชาชีพได้

\hypertarget{sec:x86_64assembly}{%
\chapter{64-bit x86 Assembly}\label{sec:x86_64assembly}}

รายวิชานี้ศึกษาทฤษฎีเกี่ยวกับสถาปัตยกรรมคอมพิวเตอร์ที่มีการพัฒนามาและทิศทางในอนาคต
ครอบคลุมเนื้อหาที่เป็นพื้นฐานที่สำคัญและจำเป็นสำหรับใช้ประกอบการศึกษาในรายวิชาต่อไป
รายวิชานี้มีการศึกษาและฝึกปฎิบัติการคอมพิวเตอร์เพื่อให้นักศึกษาได้นำทฤษฎีไปปฎิบัติใช้ได้จริง และ
เพิมพูนทักษะสำหรับเตรียมความพร้อมในการไปประกอบวิชาชีพได้

\hypertarget{sec:armv8assembly}{%
\chapter{ARMv8 Assembly}\label{sec:armv8assembly}}

รายวิชานี้ศึกษาทฤษฎีเกี่ยวกับสถาปัตยกรรมคอมพิวเตอร์ที่มีการพัฒนามาและทิศทางในอนาคต
ครอบคลุมเนื้อหาที่เป็นพื้นฐานที่สำคัญและจำเป็นสำหรับใช้ประกอบการศึกษาในรายวิชาต่อไป
รายวิชานี้มีการศึกษาและฝึกปฎิบัติการคอมพิวเตอร์เพื่อให้นักศึกษาได้นำทฤษฎีไปปฎิบัติใช้ได้จริง และ
เพิมพูนทักษะสำหรับเตรียมความพร้อมในการไปประกอบวิชาชีพได้

\hypertarget{sec:storagememorg}{%
\chapter{Storage and Memory Hierarchy}\label{sec:storagememorg}}

รายวิชานี้ศึกษาทฤษฎีเกี่ยวกับสถาปัตยกรรมคอมพิวเตอร์ที่มีการพัฒนามาและทิศทางในอนาคต
ครอบคลุมเนื้อหาที่เป็นพื้นฐานที่สำคัญและจำเป็นสำหรับใช้ประกอบการศึกษาในรายวิชาต่อไป
รายวิชานี้มีการศึกษาและฝึกปฎิบัติการคอมพิวเตอร์เพื่อให้นักศึกษาได้นำทฤษฎีไปปฎิบัติใช้ได้จริง และ
เพิมพูนทักษะสำหรับเตรียมความพร้อมในการไปประกอบวิชาชีพได้

\hypertarget{sec:codeopt}{%
\chapter{Code Optimization}\label{sec:codeopt}}

รายวิชานี้ศึกษาทฤษฎีเกี่ยวกับสถาปัตยกรรมคอมพิวเตอร์ที่มีการพัฒนามาและทิศทางในอนาคต
ครอบคลุมเนื้อหาที่เป็นพื้นฐานที่สำคัญและจำเป็นสำหรับใช้ประกอบการศึกษาในรายวิชาต่อไป
รายวิชานี้มีการศึกษาและฝึกปฎิบัติการคอมพิวเตอร์เพื่อให้นักศึกษาได้นำทฤษฎีไปปฎิบัติใช้ได้จริง และ
เพิมพูนทักษะสำหรับเตรียมความพร้อมในการไปประกอบวิชาชีพได้

\hypertarget{sec:os}{%
\chapter{The Operating System}\label{sec:os}}

อยู่ระหว่างการจัดทำ

\hypertarget{sec:sharedmem}{%
\chapter{Leveraging Shared Memory in MultiCore
Era}\label{sec:sharedmem}}

อยู่ระหว่างการจัดทำ

\hypertarget{sec:parallelsys}{%
\chapter{Other Parallel Systems}\label{sec:parallelsys}}

อยู่ระหว่างการจัดทำ

\hypertarget{appendix}{%
\chapter*{Appendix}\label{appendix}}
\addcontentsline{toc}{chapter}{Appendix}

This is the appendix.

\hypertarget{tbl:example_dataframe}{}
\begin{longtable}[]{@{}rrr@{}}
\caption{\label{tbl:example_dataframe}Example dataframe.}\tabularnewline
\toprule()
A & B & C \\
\midrule()
\endfirsthead
\toprule()
A & B & C \\
\midrule()
\endhead
1 & 1 & 1 \\
2 & 2 & 2 \\
3 & 3 & 3 \\
4 & 4 & 4 \\
5 & 5 & 5 \\
6 & 6 & 6 \\
\bottomrule()
\end{longtable}

\hypertarget{references}{%
\chapter*{References}\label{references}}
\addcontentsline{toc}{chapter}{References}

\backmatter

\end{document}
